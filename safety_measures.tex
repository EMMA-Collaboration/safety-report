\section{Safety Measures}
A number of safety measures are proposed in order to minimize the risks and effects of the hazards
described above.  The EMMA PGAC GHS is very similar to the gas handling systems used in the IRIS ionization chamber, the MAYA PPAC, and the York PGAC.
\subsection{Hardware}
\begin{enumerate}
\setlength{\itemsep}{0pt}
\setlength{\parskip}{0pt}
\setlength{\parsep}{0pt}

\item The maximum flow of isobutane from the ISAC gas shack is limited to 250\,cc/min by a low-pressure regulator feeding gas through a pre-set needle valve.
\item A flammable gas sensor wired to an alarm in the main control room will be installed near the experimental area to detected any gas
leaks.
\item To prevent air-isobutane mixtures, isobutane is provided and pumped-out using the gas system plumbing only, whereas air is provided and pumped-out using the vacuum system plumbing only. To the extent possible, this is determined by the gas system design and enforced by interlocks.
\item The procedures specified in \S\,\ref{procedures} require all gas-containing volumes to be purged of air or pumped to vacuum, and leak tested before introducing isobutane flow to the PGAC box.
%\item  Two valves---one \marnote{missing $\rightarrow$} manual and one EPICS remote---will be installed in series on all gas lines entering the chamber to safeguard against failure leading to air ingress.
%\item Flammable materials near the chamber must be limited to essential materials only.
%\item To avoid a possible source of ignition, a cold cathode vacuum gauge is used to measure the scattering chamber pressure.
\item A dedicated gas system proportional-integral-differential (PID) controller will control the isobutane flow into the PGAC box to maintain the set pressure. 
\item The HV power supplies will be operated in trip-hold mode, ensuring that, at most, only one spark can occur.
\item All valves are normally closed (N/C) types. They will close on loss of electrical power. Air-actuated valves will also close on loss of air pressure.

\end{enumerate}
\subsection{Interlocks}
\label{interlocks}
The pressure within the volume of the PGAC box will be monitored throughout the experiment. If
the volumetric flow rate changes due to the development of a leak or a burst window, interlocks will be
triggered to make the chamber safe. The specific interlocks are as follows:
\begin{enumerate}
\item \texttt{VA1} can be open if and only if \texttt{HEBT:IG19} $<$ $\textrm{P}_\textrm{max}$.

This interlock shuts off the isobutane supply if the pressure in the scattering chamber exceeds $\textrm{P}_\textrm{max}$, typically set to $7\times10^{-5}$\,Torr.  This could occur in the event of isobutane entering the  scattering chamber via a broken window or if  bypass valve\texttt{VB4}   is opened with isobutane in PGAC box, etc.
\item The HV supply can be on if and only if 
(\texttt{VA1\_open}$) \land ($\texttt{FC1} $<\textrm{Q}_\textrm{max}) \land ($\texttt{FC1} $> \textrm{Q}_\textrm{min})$

The \texttt{VA1\_open} condition ensures the potential for gas flow and ensures that the  HV supply is off when \texttt{VA1} is tripped closed. % by window breakage or \texttt{VB4} being opened.
 It also
ensures that HV is off during pumping and venting operations.

The $\textrm{Q}_\textrm{min} < $ \texttt{FC1} $ < \textrm{Q}_\textrm{max}$ condition protects against air leaking into the the PGAC box after gas flow has
been established.
If air leaks into the PGAC box, the PID will reduce \texttt{FC1} flow to keep \texttt{PA1} pressure
constant.
As long as $\textrm{Q}_\textrm{max}/\textrm{Q}_\textrm{min} \lesssim 5$ , the volumetric flow rate will fall below Q$_\textrm{min}$ and trip off the HV before a
flammable mix of air and isobutane can be formed.  For example, the leak scenario detailed in \S\ref{low_pres} requires a ratio of $\textrm{Q}_\textrm{max}/\textrm{Q}_\textrm{min} =10.9$.  Typical values are Q$_\textrm{max}=100$\,cc/min, and  Q$_\textrm{min}=20$\,cc/min.
\end{enumerate}

\noindent Additionally, the solenoid valve on the pump \texttt{PU1} automatically shuts when the pump is turned off.
\subsection{Commissioning}
Prior to the experiment, a commissioning phase will be undertaken to investigate the various
features of the system. During this period the interlocks will be tested by initiating pressure rises
using an inert gas, e.g., nitrogen. %This will enable us to set high/low pressure interlocks to within a tight margin of nominal operational pressure.
Throughout these tests, the accelerator will
be protected by a gate valve.